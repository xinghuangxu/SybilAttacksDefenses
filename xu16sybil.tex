
%% bare_conf.tex
%% V1.4b
%% 2015/08/26
%% by Michael Shell
%% See:
%% http://www.michaelshell.org/
%% for current contact information.
%%
%% This is a skeleton file demonstrating the use of IEEEtran.cls
%% (requires IEEEtran.cls version 1.8b or later) with an IEEE
%% conference paper.
%%
%% Support sites:
%% http://www.michaelshell.org/tex/ieeetran/
%% http://www.ctan.org/pkg/ieeetran
%% and
%% http://www.ieee.org/

%%*************************************************************************
%% Legal Notice:
%% This code is offered as-is without any warranty either expressed or
%% implied; without even the implied warranty of MERCHANTABILITY or
%% FITNESS FOR A PARTICULAR PURPOSE! 
%% User assumes all risk.
%% In no event shall the IEEE or any contributor to this code be liable for
%% any damages or losses, including, but not limited to, incidental,
%% consequential, or any other damages, resulting from the use or misuse
%% of any information contained here.
%%
%% All comments are the opinions of their respective authors and are not
%% necessarily endorsed by the IEEE.
%%
%% This work is distributed under the LaTeX Project Public License (LPPL)
%% ( http://www.latex-project.org/ ) version 1.3, and may be freely used,
%% distributed and modified. A copy of the LPPL, version 1.3, is included
%% in the base LaTeX documentation of all distributions of LaTeX released
%% 2003/12/01 or later.
%% Retain all contribution notices and credits.
%% ** Modified files should be clearly indicated as such, including  **
%% ** renaming them and changing author support contact information. **
%%*************************************************************************


% *** Authors should verify (and, if needed, correct) their LaTeX system  ***
% *** with the testflow diagnostic prior to trusting their LaTeX platform ***
% *** with production work. The IEEE's font choices and paper sizes can   ***
% *** trigger bugs that do not appear when using other class files.       ***                          ***
% The testflow support page is at:
% http://www.michaelshell.org/tex/testflow/



\documentclass[conference]{IEEEtran}
% Some Computer Society conferences also require the compsoc mode option,
% but others use the standard conference format.
%
% If IEEEtran.cls has not been installed into the LaTeX system files,
% manually specify the path to it like:
% \documentclass[conference]{../sty/IEEEtran}





% Some very useful LaTeX packages include:
% (uncomment the ones you want to load)


% *** MISC UTILITY PACKAGES ***
%
%\usepackage{ifpdf}
% Heiko Oberdiek's ifpdf.sty is very useful if you need conditional
% compilation based on whether the output is pdf or dvi.
% usage:
% \ifpdf
%   % pdf code
% \else
%   % dvi code
% \fi
% The latest version of ifpdf.sty can be obtained from:
% http://www.ctan.org/pkg/ifpdf
% Also, note that IEEEtran.cls V1.7 and later provides a builtin
% \ifCLASSINFOpdf conditional that works the same way.
% When switching from latex to pdflatex and vice-versa, the compiler may
% have to be run twice to clear warning/error messages.






% *** CITATION PACKAGES ***
%
%\usepackage{cite}
% cite.sty was written by Donald Arseneau
% V1.6 and later of IEEEtran pre-defines the format of the cite.sty package
% \cite{} output to follow that of the IEEE. Loading the cite package will
% result in citation numbers being automatically sorted and properly
% "compressed/ranged". e.g., [1], [9], [2], [7], [5], [6] without using
% cite.sty will become [1], [2], [5]--[7], [9] using cite.sty. cite.sty's
% \cite will automatically add leading space, if needed. Use cite.sty's
% noadjust option (cite.sty V3.8 and later) if you want to turn this off
% such as if a citation ever needs to be enclosed in parenthesis.
% cite.sty is already installed on most LaTeX systems. Be sure and use
% version 5.0 (2009-03-20) and later if using hyperref.sty.
% The latest version can be obtained at:
% http://www.ctan.org/pkg/cite
% The documentation is contained in the cite.sty file itself.






% *** GRAPHICS RELATED PACKAGES ***
%
\ifCLASSINFOpdf
  % \usepackage[pdftex]{graphicx}
  % declare the path(s) where your graphic files are
  % \graphicspath{{../pdf/}{../jpeg/}}
  % and their extensions so you won't have to specify these with
  % every instance of \includegraphics
  % \DeclareGraphicsExtensions{.pdf,.jpeg,.png}
\else
  % or other class option (dvipsone, dvipdf, if not using dvips). graphicx
  % will default to the driver specified in the system graphics.cfg if no
  % driver is specified.
  % \usepackage[dvips]{graphicx}
  % declare the path(s) where your graphic files are
  % \graphicspath{{../eps/}}
  % and their extensions so you won't have to specify these with
  % every instance of \includegraphics
  % \DeclareGraphicsExtensions{.eps}
\fi
% graphicx was written by David Carlisle and Sebastian Rahtz. It is
% required if you want graphics, photos, etc. graphicx.sty is already
% installed on most LaTeX systems. The latest version and documentation
% can be obtained at: 
% http://www.ctan.org/pkg/graphicx
% Another good source of documentation is "Using Imported Graphics in
% LaTeX2e" by Keith Reckdahl which can be found at:
% http://www.ctan.org/pkg/epslatex
%
% latex, and pdflatex in dvi mode, support graphics in encapsulated
% postscript (.eps) format. pdflatex in pdf mode supports graphics
% in .pdf, .jpeg, .png and .mps (metapost) formats. Users should ensure
% that all non-photo figures use a vector format (.eps, .pdf, .mps) and
% not a bitmapped formats (.jpeg, .png). The IEEE frowns on bitmapped formats
% which can result in "jaggedy"/blurry rendering of lines and letters as
% well as large increases in file sizes.
%
% You can find documentation about the pdfTeX application at:
% http://www.tug.org/applications/pdftex





% *** MATH PACKAGES ***
%
%\usepackage{amsmath}
% A popular package from the American Mathematical Society that provides
% many useful and powerful commands for dealing with mathematics.
%
% Note that the amsmath package sets \interdisplaylinepenalty to 10000
% thus preventing page breaks from occurring within multiline equations. Use:
%\interdisplaylinepenalty=2500
% after loading amsmath to restore such page breaks as IEEEtran.cls normally
% does. amsmath.sty is already installed on most LaTeX systems. The latest
% version and documentation can be obtained at:
% http://www.ctan.org/pkg/amsmath





% *** SPECIALIZED LIST PACKAGES ***
%
%\usepackage{algorithmic}
% algorithmic.sty was written by Peter Williams and Rogerio Brito.
% This package provides an algorithmic environment fo describing algorithms.
% You can use the algorithmic environment in-text or within a figure
% environment to provide for a floating algorithm. Do NOT use the algorithm
% floating environment provided by algorithm.sty (by the same authors) or
% algorithm2e.sty (by Christophe Fiorio) as the IEEE does not use dedicated
% algorithm float types and packages that provide these will not provide
% correct IEEE style captions. The latest version and documentation of
% algorithmic.sty can be obtained at:
% http://www.ctan.org/pkg/algorithms
% Also of interest may be the (relatively newer and more customizable)
% algorithmicx.sty package by Szasz Janos:
% http://www.ctan.org/pkg/algorithmicx




% *** ALIGNMENT PACKAGES ***
%
%\usepackage{array}
% Frank Mittelbach's and David Carlisle's array.sty patches and improves
% the standard LaTeX2e array and tabular environments to provide better
% appearance and additional user controls. As the default LaTeX2e table
% generation code is lacking to the point of almost being broken with
% respect to the quality of the end results, all users are strongly
% advised to use an enhanced (at the very least that provided by array.sty)
% set of table tools. array.sty is already installed on most systems. The
% latest version and documentation can be obtained at:
% http://www.ctan.org/pkg/array


% IEEEtran contains the IEEEeqnarray family of commands that can be used to
% generate multiline equations as well as matrices, tables, etc., of high
% quality.




% *** SUBFIGURE PACKAGES ***
%\ifCLASSOPTIONcompsoc
%  \usepackage[caption=false,font=normalsize,labelfont=sf,textfont=sf]{subfig}
%\else
%  \usepackage[caption=false,font=footnotesize]{subfig}
%\fi
% subfig.sty, written by Steven Douglas Cochran, is the modern replacement
% for subfigure.sty, the latter of which is no longer maintained and is
% incompatible with some LaTeX packages including fixltx2e. However,
% subfig.sty requires and automatically loads Axel Sommerfeldt's caption.sty
% which will override IEEEtran.cls' handling of captions and this will result
% in non-IEEE style figure/table captions. To prevent this problem, be sure
% and invoke subfig.sty's "caption=false" package option (available since
% subfig.sty version 1.3, 2005/06/28) as this is will preserve IEEEtran.cls
% handling of captions.
% Note that the Computer Society format requires a larger sans serif font
% than the serif footnote size font used in traditional IEEE formatting
% and thus the need to invoke different subfig.sty package options depending
% on whether compsoc mode has been enabled.
%
% The latest version and documentation of subfig.sty can be obtained at:
% http://www.ctan.org/pkg/subfig




% *** FLOAT PACKAGES ***
%
%\usepackage{fixltx2e}
% fixltx2e, the successor to the earlier fix2col.sty, was written by
% Frank Mittelbach and David Carlisle. This package corrects a few problems
% in the LaTeX2e kernel, the most notable of which is that in current
% LaTeX2e releases, the ordering of single and double column floats is not
% guaranteed to be preserved. Thus, an unpatched LaTeX2e can allow a
% single column figure to be placed prior to an earlier double column
% figure.
% Be aware that LaTeX2e kernels dated 2015 and later have fixltx2e.sty's
% corrections already built into the system in which case a warning will
% be issued if an attempt is made to load fixltx2e.sty as it is no longer
% needed.
% The latest version and documentation can be found at:
% http://www.ctan.org/pkg/fixltx2e


%\usepackage{stfloats}
% stfloats.sty was written by Sigitas Tolusis. This package gives LaTeX2e
% the ability to do double column floats at the bottom of the page as well
% as the top. (e.g., "\begin{figure*}[!b]" is not normally possible in
% LaTeX2e). It also provides a command:
%\fnbelowfloat
% to enable the placement of footnotes below bottom floats (the standard
% LaTeX2e kernel puts them above bottom floats). This is an invasive package
% which rewrites many portions of the LaTeX2e float routines. It may not work
% with other packages that modify the LaTeX2e float routines. The latest
% version and documentation can be obtained at:
% http://www.ctan.org/pkg/stfloats
% Do not use the stfloats baselinefloat ability as the IEEE does not allow
% \baselineskip to stretch. Authors submitting work to the IEEE should note
% that the IEEE rarely uses double column equations and that authors should try
% to avoid such use. Do not be tempted to use the cuted.sty or midfloat.sty
% packages (also by Sigitas Tolusis) as the IEEE does not format its papers in
% such ways.
% Do not attempt to use stfloats with fixltx2e as they are incompatible.
% Instead, use Morten Hogholm'a dblfloatfix which combines the features
% of both fixltx2e and stfloats:
%
% \usepackage{dblfloatfix}
% The latest version can be found at:
% http://www.ctan.org/pkg/dblfloatfix




% *** PDF, URL AND HYPERLINK PACKAGES ***
%
%\usepackage{url}
% url.sty was written by Donald Arseneau. It provides better support for
% handling and breaking URLs. url.sty is already installed on most LaTeX
% systems. The latest version and documentation can be obtained at:
% http://www.ctan.org/pkg/url
% Basically, \url{my_url_here}.




% *** Do not adjust lengths that control margins, column widths, etc. ***
% *** Do not use packages that alter fonts (such as pslatex).         ***
% There should be no need to do such things with IEEEtran.cls V1.6 and later.
% (Unless specifically asked to do so by the journal or conference you plan
% to submit to, of course. )


% correct bad hyphenation here
\hyphenation{op-tical net-works semi-conduc-tor}


\begin{document}
%
% paper title
% Titles are generally capitalized except for words such as a, an, and, as,
% at, but, by, for, in, nor, of, on, or, the, to and up, which are usually
% not capitalized unless they are the first or last word of the title.
% Linebreaks \\ can be used within to get better formatting as desired.
% Do not put math or special symbols in the title.
\title{A Survey on Sybil Attacks and Defenses}


% author names and affiliations
% use a multiple column layout for up to three different
% affiliations
\author{\IEEEauthorblockN{Xinghuang Xu}
\IEEEauthorblockA{EECS Department\\
Wichita State University\\
Email: xxxu3@wichita.edu}}


% conference papers do not typically use \thanks and this command
% is locked out in conference mode. If really needed, such as for
% the acknowledgment of grants, issue a \IEEEoverridecommandlockouts
% after \documentclass

% for over three affiliations, or if they all won't fit within the width
% of the page, use this alternative format:
% 
%\author{\IEEEauthorblockN{Michael Shell\IEEEauthorrefmark{1},
%Homer Simpson\IEEEauthorrefmark{2},
%James Kirk\IEEEauthorrefmark{3}, 
%Montgomery Scott\IEEEauthorrefmark{3} and
%Eldon Tyrell\IEEEauthorrefmark{4}}
%\IEEEauthorblockA{\IEEEauthorrefmark{1}School of Electrical and Computer Engineering\\
%Georgia Institute of Technology,
%Atlanta, Georgia 30332--0250\\ Email: see http://www.michaelshell.org/contact.html}
%\IEEEauthorblockA{\IEEEauthorrefmark{2}Twentieth Century Fox, Springfield, USA\\
%Email: homer@thesimpsons.com}
%\IEEEauthorblockA{\IEEEauthorrefmark{3}Starfleet Academy, San Francisco, California 96678-2391\\
%Telephone: (800) 555--1212, Fax: (888) 555--1212}
%\IEEEauthorblockA{\IEEEauthorrefmark{4}Tyrell Inc., 123 Replicant Street, Los Angeles, California 90210--4321}}




% use for special paper notices
%\IEEEspecialpapernotice{(Invited Paper)}




% make the title area
\maketitle

% As a general rule, do not put math, special symbols or citations
% in the abstract
\begin{abstract}
This  proposal presents sybil attacks under different context and categorized various sybil defense mechanisms. 
We briefly describe the main idea behind each defense categories and outline what to expect in the final survey paper. 
\end{abstract}

% no keywords




% For peer review papers, you can put extra information on the cover
% page as needed:
% \ifCLASSOPTIONpeerreview
% \begin{center} \bfseries EDICS Category: 3-BBND \end{center}
% \fi
%
% For peerreview papers, this IEEEtran command inserts a page break and
% creates the second title. It will be ignored for other modes.
\IEEEpeerreviewmaketitle



\section{Introduction}
% no \IEEEPARstart

Sybil attacks have always been a thread to distributed systems.
In a Sybil attack, an adversary will try to generate as many identities/sybils as possible and act as if he/she is multiple nodes in a network graph in order to disrupt the proper functionality of the targeted system. 
There are many different types of sybil attacks under different contexts with different goals. 
The goal of an attacker launching a sybil attack might vary. For example, it can be to distupt a distributed netework, to influence a distributed recommender system, to gain an unfair amount of share resource or to gather valuable information.
We will go in more detail of how sybil attacks can achieve the above goals. Apart from listing many different kinds of sybil attacks, defenses are also presented.
Traditionally, some people have classified sybil defenses based on whether the solution is centralized or distributed.
Others have classified them based on their underlying technologies. In this paper, we propose an innovative classification approach. 
A defense is first grouped as traditional sybil defense, domain specific defense or social network graph based defense, then it will be futher classified based on the underlying technologies used. The target audience of the final survey is people who are interested in Sybil attacks or who maintain a distributed system and want to deploy a Sybil attack defense mechanism in their system.

In the following sections, we will list different sybil attacks and various denfenses. Afterward, future work for the final survey is outlined.

\section{Sybil Attacks}
% no \IEEEPARstart

\subsection{Routing}
In a distributed system, routing required the participation of many nodes. For example, when node A wants to look up node D, it will ask its neighbors and its neighbors would go ask their neighbors and on and on until someone has the knowledge of D and pass D’s location all the way back to A through a path. An attacker can inject many malicious nodes inside and network to disrupt the routing. In a sybil infected network, when A ask about D’s location, if the lookup path pass a malicious node, the malicious node can return a false address or can do nothing. Either way, the process of lookup will be slow down greatly. With sufficient proportions of sybil nodes, attackers can block communications among nodes altogether and render the system useless.

There are concrete real world examples of Sybil attacks on routing protocols.
Geogaphical routing protocol is vulnerable to Sybil attacks because it requires nodes to exchange coordinate data with their neighbors to efficiently address packets. By using the Sybil attack, an attacker can create multiple identities in different geographical locations thus making him available in multiple places at once which violates the fundamental assumption of the routing protocol\cite{Karlof03securerouting}. Sybil attack pose a thread to the seemingly robust multipath routing protocol too.\cite{Karlof03securerouting}

The popular Distributed Hash Tables (DHT) which underlies many peer-to-peer systems are also known to be vulnerable to Sybil attack. In DHT that are opened to the rest of the world like Vuze DHT, an adversary can introduce a large number of corrupt nodes in the network to degrade of the performance of the targeted DHT.\cite{Danezis05sybil-resistantdht} 

\subsection{Content Rating System}
Sybil attack is a fundamental thread to any user-based content rating system such as Goodreads, Youtube and IMDB. There are hugh incentives in this kind of attacks because attackers can promote low-quality content to a wide audience. For example, it has been studied that many people check the IMDB score before going to see the movie in theater. A high IMDB score will attrack more audiences to go to the theater thus making the movie more profitable. This is not hypertheticcal. There are successful real world cases. For example, the famous Slashdot poll on the best compute science school has caused students to write automatic scripts to vote for their schools repeatedly. Moreover, some underground companies made money through assisting clients in promoting their Youtube video's view counts by using a large number of Sybil accounts.\cite{Tran09SOC}

\subsection{Reputation System}
Sybil attack post a significant challenge for building repuations systems. In a repuation system, an adversary can create a large number of identities and maliciously increase the reputation of one or more master identities by giving false recommendations to them. Sybils can also promote their own reputations and falsely accuse well-behaved players in the system to hurt their reputation. For example, in eBay.com repuation is calculated as the sum of (+1,0.-1) of all the trasaction ratings no matter how big the transaction is. Sybils can be create to make small transactions with a seller and automattically give them good reviews to boost their reputation. Afterward, the seller can use that repuation on a dishonest transaction of high value. By using Sybil attack, a dishonest seller can hide the fact he frequently misbehaves at a certain rate.

Moreover, in networks that use repuation scheme to find misbehaving nodes/sybils, nodes with good reputation can report nodes they believe to be misbehaving in its neighbors. But this scheme can backfire.  For example, users can collude to artificially boost the reputation values of one or more friends, or falsely accuse well-behaved users of misbehavior. when adversaries control enough nodes and decide to repeatedly report honest nodes. The outcome is that most of the honest nodes will be considered malicious and be removed from the networks, the malicious nodes afterward will take full control of the whole system and use it for their their own benefits. Detecting such collusion attacks is yet an unsolved problem that severely limits the impact of existing reputation systems.\cite{Swamynathan10reputation}\cite{Lian07anempirical}


\subsection{Other Contexts}
There are many other forms of Sybil Attacks. It can be used to steal more share resrouces.
In a distributed system, people are sharing their resources such as bandwidth, memory and data. An adversary can create sybils to claim an unfair and disproportionate share of the resources. 

Also in distributed storage, sybils can cause data lost by being selfish and not storing the fragment of data that are asked to store. Sybils can be used to degrade the performance of the distributed file system by not responding to file request or provide the wrong file segment. What's more, because some file systems replicate data to neighbor nodes, sybils can be used to crawl the entire file system through frequently hopping into different areas in the network.\cite{Lian07anempirical}\cite{wolchok09defeatingvanish}


\section{Sybil Defenses}
% no \IEEEPARstart
The problem of defending against sybil attacks has been thoroughly study but there are no good known method that could completely eliminate the problem. Some of the centralized solution claim to be able to eliminate Sybil attacks at the price of adding authentication overhead to the system or sacrifising the open nature of distributed system. Most of the approaches study in our survey are seeking to reduce the effect of Sybil attacks in their systems instead of eliminating them. There is always a trade of between efficiency, security and system compelxity in all the approaches. In this section, we first classify the defense based on its timeline and then on the underlying technique they use. We include a short description for each teachnique which will be extended in the final paper.

\subsection{Traditional Appraoch}

\subsubsection{Trusted Certification}
This is the most popular solution for countering Sybil attacks, it required a trusted certifying authority that validates the identity of a node before it joins the system. There are two variations in this approach. One is the centralized version, the other is the semi-centralized version. In the centralized version, it is assumped that there is a trusted central authority who can verify the validity of each participant. After the validation, a certificate will be given to each participant. The participant then can use the certificate to access the system. The model is very popular and has been used widely. Most authentication services use this kind of model. The semi-centralized approach seek to cut of the cost of asymmetric criptography used in the centralized version. It leverage a techniqe called partical identity verifications. The approach still need to rely on a trusted base station but reduce the involvement of a third party authority.

The problem of  trusted certification approach is that it rely on a centralized trusted authority for credential generation, assignment and verification. However, it sacrifice the open nature that underlies the success of these distributed systems and increase the overhead of the system. \cite{newsome04sybil}\cite{Castro02Secure}\cite{Adya02FFA}.

\subsubsection{Resource Testing}
Resource Testing is another line of solution. The idea behind resource testing is that each identity should own a fair amount of resource because it runs on a legitimate cilent otherwise there is a high potential that this is a sybil node. The question is how can we test that there are resource backing up a node? Some propose the testing of IP address because multiple identities sharing a single IP address is a good sign of Sybil attacks. Others test resources such as computing power, network bandwidth. A variation of the resource testing method is called “Recurring Costs”. For example, in one solution participants are required to perform some tasks such as solving puzzles\cite{Borisov06CPS} periodically. Turing tests are also suggested as a recurring cost solution\cite{Ahn03CUH}. With Recurring Costs, the cost of Sybil attacks have become more expensive but would the benefit still outweight the cost? Cloud services have definitely help drive down the cost of Sybil attacks.

\subsection{Domain Specific Approaches}
\subsubsection{Ad hoc Networks}

In wireless ad hoc networks, a group of sybils are usually sharing the same device and they can be detected through monitoring signals’ features or the moving patterns of coexisting identities. SybilCast is a novoel protocol proposed in \cite{Zheng_thwartingsybil} that can limit the number of 
fake identities in centralized multichannel wireless networks. SybilCast can ensure that each honest user gets at least a constant fraction of their fair share of the bandwidth and complete his or her data download in asymptotically optimal time.

\subsubsection{Wireless Sensor Netowrk}
In wireless sensor netowrk, Demirbas et al. proposed an sybil attack counter measurement by using received signal strength indicator (RSSI). The algorithm proposed in \cite{Demirbas06RSSI} claims to be light weight because it only require the colloboration of one other node apart from the receiver and accurate because it detects sybil attack cases with 100\% completeness and only a few percent false positives.
\cite{Demirbas06RSSI}

\subsection{Social Network-Based}
Yu et al. has started a new era of sybil defense when he proposed the idea of detecting sybils using a unique structure in the social network graph. Even though attackers can inject many sybils into a social graph, the connections between honest users and sybils are limited\cite{Yu08SybilGuard}. For example, honest users on facebook would not randomly accept friends if they do not know the person. Suprisingly, the social network appraoch has showed to be able to overcome some of the earlier approaches limitations and shortcomings.

\subsubsection{SybilGuard}
SybilGuard designed by Yu et al. \cite{Yu08SybilGuard} is one of the first Sybil defense techniques based on Social Network. The approach assumes that each edge in the graph between two identities indicates a human-established trust relationship and malicious users can only create limited edges between honest users. SybilGuard bounds the number of malicious sybils a user can create by exploiting the property that there exist a disproportionaltely small "cut" in the 
graph between the sybil nodes and the honest nodes.

\subsubsection{SybilLimit}
The approach take by SybilLimit in \cite{Yu08SybilLimit} is the same as SybilGuard but SybilGuard can dramatically reduce the number of sybil nodes accepted by a factor of  $ \sqrt{n} $.

\subsubsection{SybilInfer}
SybilInfer takes the approach of labelling nodes in a social network as honest users or Sybils. Internally, it uses a probabilistic model of honest social networks as its knowledge base and an inference engine to obtain the potential regions of dishonest nodes.
It claims to be more accurate and more applicable when compare to both SybilGuard and SybilLimit.


\section{Project Deliverable}
% no \IEEEPARstart
We have listed potential threads of Sybil attacks under different context and showed different types of counter measurements. 
We want the final paper to serve as introduction and guidline for sybil attacks and defenses.
To obtain this goal, we would like to present more detail explaintion of each Sybil defense approaches in the final survey. Furthermore, a comphrehensive comparison among the different defense mechanisms will be included.
If time permits, we would also look into evaluating some of the defense mechanisms by using metrics like false negative/positive rate, the time and code complexity the soution would add to the existing system.
Also we would like to point out potential directions/opportunities in the research of sybil attacks/defenses in the final survey paper.

% An example of a floating figure using the graphicx package.
% Note that \label must occur AFTER (or within) \caption.
% For figures, \caption should occur after the \includegraphics.
% Note that IEEEtran v1.7 and later has special internal code that
% is designed to preserve the operation of \label within \caption
% even when the captionsoff option is in effect. However, because
% of issues like this, it may be the safest practice to put all your
% \label just after \caption rather than within \caption{}.
%
% Reminder: the "draftcls" or "draftclsnofoot", not "draft", class
% option should be used if it is desired that the figures are to be
% displayed while in draft mode.
%
%\begin{figure}[!t]
%\centering
%\includegraphics[width=2.5in]{myfigure}
% where an .eps filename suffix will be assumed under latex, 
% and a .pdf suffix will be assumed for pdflatex; or what has been declared
% via \DeclareGraphicsExtensions.
%\caption{Simulation results for the network.}
%\label{fig_sim}
%\end{figure}

% Note that the IEEE typically puts floats only at the top, even when this
% results in a large percentage of a column being occupied by floats.


% An example of a double column floating figure using two subfigures.
% (The subfig.sty package must be loaded for this to work.)
% The subfigure \label commands are set within each subfloat command,
% and the \label for the overall figure must come after \caption.
% \hfil is used as a separator to get equal spacing.
% Watch out that the combined width of all the subfigures on a 
% line do not exceed the text width or a line break will occur.
%
%\begin{figure*}[!t]
%\centering
%\subfloat[Case I]{\includegraphics[width=2.5in]{box}%
%\label{fig_first_case}}
%\hfil
%\subfloat[Case II]{\includegraphics[width=2.5in]{box}%
%\label{fig_second_case}}
%\caption{Simulation results for the network.}
%\label{fig_sim}
%\end{figure*}
%
% Note that often IEEE papers with subfigures do not employ subfigure
% captions (using the optional argument to \subfloat[]), but instead will
% reference/describe all of them (a), (b), etc., within the main caption.
% Be aware that for subfig.sty to generate the (a), (b), etc., subfigure
% labels, the optional argument to \subfloat must be present. If a
% subcaption is not desired, just leave its contents blank,
% e.g., \subfloat[].


% An example of a floating table. Note that, for IEEE style tables, the
% \caption command should come BEFORE the table and, given that table
% captions serve much like titles, are usually capitalized except for words
% such as a, an, and, as, at, but, by, for, in, nor, of, on, or, the, to
% and up, which are usually not capitalized unless they are the first or
% last word of the caption. Table text will default to \footnotesize as
% the IEEE normally uses this smaller font for tables.
% The \label must come after \caption as always.
%
%\begin{table}[!t]
%% increase table row spacing, adjust to taste
%\renewcommand{\arraystretch}{1.3}
% if using array.sty, it might be a good idea to tweak the value of
% \extrarowheight as needed to properly center the text within the cells
%\caption{An Example of a Table}
%\label{table_example}
%\centering
%% Some packages, such as MDW tools, offer better commands for making tables
%% than the plain LaTeX2e tabular which is used here.
%\begin{tabular}{|c||c|}
%\hline
%One & Two\\
%\hline
%Three & Four\\
%\hline
%\end{tabular}
%\end{table}


% Note that the IEEE does not put floats in the very first column
% - or typically anywhere on the first page for that matter. Also,
% in-text middle ("here") positioning is typically not used, but it
% is allowed and encouraged for Computer Society conferences (but
% not Computer Society journals). Most IEEE journals/conferences use
% top floats exclusively. 
% Note that, LaTeX2e, unlike IEEE journals/conferences, places
% footnotes above bottom floats. This can be corrected via the
% \fnbelowfloat command of the stfloats package.


% trigger a \newpage just before the given reference
% number - used to balance the columns on the last page
% adjust value as needed - may need to be readjusted if
% the document is modified later
%\IEEEtriggeratref{8}
% The "triggered" command can be changed if desired:
%\IEEEtriggercmd{\enlargethispage{-5in}}

% references section

% can use a bibliography generated by BibTeX as a .bbl file
% BibTeX documentation can be easily obtained at:
% http://mirror.ctan.org/biblio/bibtex/contrib/doc/
% The IEEEtran BibTeX style support page is at:
% http://www.michaelshell.org/tex/ieeetran/bibtex/
%\bibliographystyle{IEEEtran}
% argument is your BibTeX string definitions and bibliography database(s)
%\bibliography{IEEEabrv,../bib/paper}
%
% <OR> manually copy in the resultant .bbl file
% set second argument of \begin to the number of references
% (used to reserve space for the reference number labels box)

\bibliography{diy} 
\bibliographystyle{ieeetr}


% that's all folks
\end{document}


