
%% bare_conf.tex
%% V1.4b
%% 2015/08/26
%% by Michael Shell
%% See:
%% http://www.michaelshell.org/
%% for current contact information.
%%
%% This is a skeleton file demonstrating the use of IEEEtran.cls
%% (requires IEEEtran.cls version 1.8b or later) with an IEEE
%% conference paper.
%%
%% Support sites:
%% http://www.michaelshell.org/tex/ieeetran/
%% http://www.ctan.org/pkg/ieeetran
%% and
%% http://www.ieee.org/

%%*************************************************************************
%% Legal Notice:
%% This code is offered as-is without any warranty either expressed or
%% implied; without even the implied warranty of MERCHANTABILITY or
%% FITNESS FOR A PARTICULAR PURPOSE! 
%% User assumes all risk.
%% In no event shall the IEEE or any contributor to this code be liable for
%% any damages or losses, including, but not limited to, incidental,
%% consequential, or any other damages, resulting from the use or misuse
%% of any information contained here.
%%
%% All comments are the opinions of their respective authors and are not
%% necessarily endorsed by the IEEE.
%%
%% This work is distributed under the LaTeX Project Public License (LPPL)
%% ( http://www.latex-project.org/ ) version 1.3, and may be freely used,
%% distributed and modified. A copy of the LPPL, version 1.3, is included
%% in the base LaTeX documentation of all distributions of LaTeX released
%% 2003/12/01 or later.
%% Retain all contribution notices and credits.
%% ** Modified files should be clearly indicated as such, including  **
%% ** renaming them and changing author support contact information. **
%%*************************************************************************


% *** Authors should verify (and, if needed, correct) their LaTeX system  ***
% *** with the testflow diagnostic prior to trusting their LaTeX platform ***
% *** with production work. The IEEE's font choices and paper sizes can   ***
% *** trigger bugs that do not appear when using other class files.       ***                          ***
% The testflow support page is at:
% http://www.michaelshell.org/tex/testflow/



\documentclass[conference]{IEEEtran}
% Some Computer Society conferences also require the compsoc mode option,
% but others use the standard conference format.
%
% If IEEEtran.cls has not been installed into the LaTeX system files,
% manually specify the path to it like:
% \documentclass[conference]{../sty/IEEEtran}





% Some very useful LaTeX packages include:
% (uncomment the ones you want to load)


% *** MISC UTILITY PACKAGES ***
%
%\usepackage{ifpdf}
% Heiko Oberdiek's ifpdf.sty is very useful if you need conditional
% compilation based on whether the output is pdf or dvi.
% usage:
% \ifpdf
%   % pdf code
% \else
%   % dvi code
% \fi
% The latest version of ifpdf.sty can be obtained from:
% http://www.ctan.org/pkg/ifpdf
% Also, note that IEEEtran.cls V1.7 and later provides a builtin
% \ifCLASSINFOpdf conditional that works the same way.
% When switching from latex to pdflatex and vice-versa, the compiler may
% have to be run twice to clear warning/error messages.






% *** CITATION PACKAGES ***
%
%\usepackage{cite}
% cite.sty was written by Donald Arseneau
% V1.6 and later of IEEEtran pre-defines the format of the cite.sty package
% \cite{} output to follow that of the IEEE. Loading the cite package will
% result in citation numbers being automatically sorted and properly
% "compressed/ranged". e.g., [1], [9], [2], [7], [5], [6] without using
% cite.sty will become [1], [2], [5]--[7], [9] using cite.sty. cite.sty's
% \cite will automatically add leading space, if needed. Use cite.sty's
% noadjust option (cite.sty V3.8 and later) if you want to turn this off
% such as if a citation ever needs to be enclosed in parenthesis.
% cite.sty is already installed on most LaTeX systems. Be sure and use
% version 5.0 (2009-03-20) and later if using hyperref.sty.
% The latest version can be obtained at:
% http://www.ctan.org/pkg/cite
% The documentation is contained in the cite.sty file itself.






% *** GRAPHICS RELATED PACKAGES ***
%
\ifCLASSINFOpdf
  % \usepackage[pdftex]{graphicx}
  % declare the path(s) where your graphic files are
  % \graphicspath{{../pdf/}{../jpeg/}}
  % and their extensions so you won't have to specify these with
  % every instance of \includegraphics
  % \DeclareGraphicsExtensions{.pdf,.jpeg,.png}
\else
  % or other class option (dvipsone, dvipdf, if not using dvips). graphicx
  % will default to the driver specified in the system graphics.cfg if no
  % driver is specified.
  % \usepackage[dvips]{graphicx}
  % declare the path(s) where your graphic files are
  % \graphicspath{{../eps/}}
  % and their extensions so you won't have to specify these with
  % every instance of \includegraphics
  % \DeclareGraphicsExtensions{.eps}
\fi
% graphicx was written by David Carlisle and Sebastian Rahtz. It is
% required if you want graphics, photos, etc. graphicx.sty is already
% installed on most LaTeX systems. The latest version and documentation
% can be obtained at: 
% http://www.ctan.org/pkg/graphicx
% Another good source of documentation is "Using Imported Graphics in
% LaTeX2e" by Keith Reckdahl which can be found at:
% http://www.ctan.org/pkg/epslatex
%
% latex, and pdflatex in dvi mode, support graphics in encapsulated
% postscript (.eps) format. pdflatex in pdf mode supports graphics
% in .pdf, .jpeg, .png and .mps (metapost) formats. Users should ensure
% that all non-photo figures use a vector format (.eps, .pdf, .mps) and
% not a bitmapped formats (.jpeg, .png). The IEEE frowns on bitmapped formats
% which can result in "jaggedy"/blurry rendering of lines and letters as
% well as large increases in file sizes.
%
% You can find documentation about the pdfTeX application at:
% http://www.tug.org/applications/pdftex





% *** MATH PACKAGES ***
%
%\usepackage{amsmath}
% A popular package from the American Mathematical Society that provides
% many useful and powerful commands for dealing with mathematics.
%
% Note that the amsmath package sets \interdisplaylinepenalty to 10000
% thus preventing page breaks from occurring within multiline equations. Use:
%\interdisplaylinepenalty=2500
% after loading amsmath to restore such page breaks as IEEEtran.cls normally
% does. amsmath.sty is already installed on most LaTeX systems. The latest
% version and documentation can be obtained at:
% http://www.ctan.org/pkg/amsmath





% *** SPECIALIZED LIST PACKAGES ***
%
%\usepackage{algorithmic}
% algorithmic.sty was written by Peter Williams and Rogerio Brito.
% This package provides an algorithmic environment fo describing algorithms.
% You can use the algorithmic environment in-text or within a figure
% environment to provide for a floating algorithm. Do NOT use the algorithm
% floating environment provided by algorithm.sty (by the same authors) or
% algorithm2e.sty (by Christophe Fiorio) as the IEEE does not use dedicated
% algorithm float types and packages that provide these will not provide
% correct IEEE style captions. The latest version and documentation of
% algorithmic.sty can be obtained at:
% http://www.ctan.org/pkg/algorithms
% Also of interest may be the (relatively newer and more customizable)
% algorithmicx.sty package by Szasz Janos:
% http://www.ctan.org/pkg/algorithmicx




% *** ALIGNMENT PACKAGES ***
%
%\usepackage{array}
% Frank Mittelbach's and David Carlisle's array.sty patches and improves
% the standard LaTeX2e array and tabular environments to provide better
% appearance and additional user controls. As the default LaTeX2e table
% generation code is lacking to the point of almost being broken with
% respect to the quality of the end results, all users are strongly
% advised to use an enhanced (at the very least that provided by array.sty)
% set of table tools. array.sty is already installed on most systems. The
% latest version and documentation can be obtained at:
% http://www.ctan.org/pkg/array


% IEEEtran contains the IEEEeqnarray family of commands that can be used to
% generate multiline equations as well as matrices, tables, etc., of high
% quality.




% *** SUBFIGURE PACKAGES ***
%\ifCLASSOPTIONcompsoc
%  \usepackage[caption=false,font=normalsize,labelfont=sf,textfont=sf]{subfig}
%\else
%  \usepackage[caption=false,font=footnotesize]{subfig}
%\fi
% subfig.sty, written by Steven Douglas Cochran, is the modern replacement
% for subfigure.sty, the latter of which is no longer maintained and is
% incompatible with some LaTeX packages including fixltx2e. However,
% subfig.sty requires and automatically loads Axel Sommerfeldt's caption.sty
% which will override IEEEtran.cls' handling of captions and this will result
% in non-IEEE style figure/table captions. To prevent this problem, be sure
% and invoke subfig.sty's "caption=false" package option (available since
% subfig.sty version 1.3, 2005/06/28) as this is will preserve IEEEtran.cls
% handling of captions.
% Note that the Computer Society format requires a larger sans serif font
% than the serif footnote size font used in traditional IEEE formatting
% and thus the need to invoke different subfig.sty package options depending
% on whether compsoc mode has been enabled.
%
% The latest version and documentation of subfig.sty can be obtained at:
% http://www.ctan.org/pkg/subfig




% *** FLOAT PACKAGES ***
%
%\usepackage{fixltx2e}
% fixltx2e, the successor to the earlier fix2col.sty, was written by
% Frank Mittelbach and David Carlisle. This package corrects a few problems
% in the LaTeX2e kernel, the most notable of which is that in current
% LaTeX2e releases, the ordering of single and double column floats is not
% guaranteed to be preserved. Thus, an unpatched LaTeX2e can allow a
% single column figure to be placed prior to an earlier double column
% figure.
% Be aware that LaTeX2e kernels dated 2015 and later have fixltx2e.sty's
% corrections already built into the system in which case a warning will
% be issued if an attempt is made to load fixltx2e.sty as it is no longer
% needed.
% The latest version and documentation can be found at:
% http://www.ctan.org/pkg/fixltx2e


%\usepackage{stfloats}
% stfloats.sty was written by Sigitas Tolusis. This package gives LaTeX2e
% the ability to do double column floats at the bottom of the page as well
% as the top. (e.g., "\begin{figure*}[!b]" is not normally possible in
% LaTeX2e). It also provides a command:
%\fnbelowfloat
% to enable the placement of footnotes below bottom floats (the standard
% LaTeX2e kernel puts them above bottom floats). This is an invasive package
% which rewrites many portions of the LaTeX2e float routines. It may not work
% with other packages that modify the LaTeX2e float routines. The latest
% version and documentation can be obtained at:
% http://www.ctan.org/pkg/stfloats
% Do not use the stfloats baselinefloat ability as the IEEE does not allow
% \baselineskip to stretch. Authors submitting work to the IEEE should note
% that the IEEE rarely uses double column equations and that authors should try
% to avoid such use. Do not be tempted to use the cuted.sty or midfloat.sty
% packages (also by Sigitas Tolusis) as the IEEE does not format its papers in
% such ways.
% Do not attempt to use stfloats with fixltx2e as they are incompatible.
% Instead, use Morten Hogholm'a dblfloatfix which combines the features
% of both fixltx2e and stfloats:
%
% \usepackage{dblfloatfix}
% The latest version can be found at:
% http://www.ctan.org/pkg/dblfloatfix




% *** PDF, URL AND HYPERLINK PACKAGES ***
%
%\usepackage{url}
% url.sty was written by Donald Arseneau. It provides better support for
% handling and breaking URLs. url.sty is already installed on most LaTeX
% systems. The latest version and documentation can be obtained at:
% http://www.ctan.org/pkg/url
% Basically, \url{my_url_here}.




% *** Do not adjust lengths that control margins, column widths, etc. ***
% *** Do not use packages that alter fonts (such as pslatex).         ***
% There should be no need to do such things with IEEEtran.cls V1.6 and later.
% (Unless specifically asked to do so by the journal or conference you plan
% to submit to, of course. )


% correct bad hyphenation here
\hyphenation{op-tical net-works semi-conduc-tor}


\begin{document}
%
% paper title
% Titles are generally capitalized except for words such as a, an, and, as,
% at, but, by, for, in, nor, of, on, or, the, to and up, which are usually
% not capitalized unless they are the first or last word of the title.
% Linebreaks \\ can be used within to get better formatting as desired.
% Do not put math or special symbols in the title.
\title{A Survey on Sybil Attacks and Defenses}


% author names and affiliations
% use a multiple column layout for up to three different
% affiliations
\author{\IEEEauthorblockN{Xinghuang Xu}
\IEEEauthorblockA{EECS Department\\
Wichita State University\\
Email: xxxu3@wichita.edu}}


% conference papers do not typically use \thanks and this command
% is locked out in conference mode. If really needed, such as for
% the acknowledgment of grants, issue a \IEEEoverridecommandlockouts
% after \documentclass

% for over three affiliations, or if they all won't fit within the width
% of the page, use this alternative format:
% 
%\author{\IEEEauthorblockN{Michael Shell\IEEEauthorrefmark{1},
%Homer Simpson\IEEEauthorrefmark{2},
%James Kirk\IEEEauthorrefmark{3}, 
%Montgomery Scott\IEEEauthorrefmark{3} and
%Eldon Tyrell\IEEEauthorrefmark{4}}
%\IEEEauthorblockA{\IEEEauthorrefmark{1}School of Electrical and Computer Engineering\\
%Georgia Institute of Technology,
%Atlanta, Georgia 30332--0250\\ Email: see http://www.michaelshell.org/contact.html}
%\IEEEauthorblockA{\IEEEauthorrefmark{2}Twentieth Century Fox, Springfield, USA\\
%Email: homer@thesimpsons.com}
%\IEEEauthorblockA{\IEEEauthorrefmark{3}Starfleet Academy, San Francisco, California 96678-2391\\
%Telephone: (800) 555--1212, Fax: (888) 555--1212}
%\IEEEauthorblockA{\IEEEauthorrefmark{4}Tyrell Inc., 123 Replicant Street, Los Angeles, California 90210--4321}}




% use for special paper notices
%\IEEEspecialpapernotice{(Invited Paper)}




% make the title area
\maketitle

% As a general rule, do not put math, special symbols or citations
% in the abstract
\begin{abstract}
Sybil attack has been a thread to most peer to peer network systems. 
If new identities can be created without control in a distributed system, the system is susceptible to Sybil attacks.
For example in IMDB, sybil accounts can be created to boost the score of a new movie in order to attract potential audiences to watch the movie in theatres.
The survey paper is a guideline for open distributed system designers who want to introduce defense mechanisms into their systems to protect against Sybil attacks.
We first define various Sybil attacks under different domains with different goals then we presents three categories of defenses against Sybil attacks.
The three categories include the traditional approach, the social network based approach and the domain specific approach.
We also analyse the differences among the three categories and the advantage/disadvantage of methods within each category. 
Readers will have a deeper understanding of how to protect their distributed systems against Sybil attacks after reading this survey.

\end{abstract}

% no keywords




% For peer review papers, you can put extra information on the cover
% page as needed:
% \ifCLASSOPTIONpeerreview
% \begin{center} \bfseries EDICS Category: 3-BBND \end{center}
% \fi
%
% For peerreview papers, this IEEEtran command inserts a page break and
% creates the second title. It will be ignored for other modes.
\IEEEpeerreviewmaketitle



\section{Introduction}

Sybil is book written by Flora Rheta Schreiber about the treatment of Sybil Dorsett for dissociative identity disorder. She is believed to have manifested sixteen different
personalities according her doctor Cornelia B Wilbur.\cite{sybil16wiki}

This survey is about a specific system security attack name Sybil Attack. Sybil attack takes place when an adversary in a system acts as if he is multiple users
with different identities in order to disrupt the proper functionality of the underlying system and benefit himself. Sybil attack has proven to be harmful in many systems. 
For example, in a book recommender system.
The popularity of a book depends on the number of people who have liked it. In such a system, the goal is to find books that is likely to be of interest to users based on others'
recommendations. An attacker can create many fake accounts and out vote legitimate users on the books he/she wants to promote or demote. The success of the attack is almost guaranteed given
the fact that most legitimate users are too lazy to vote in a recommender system. 
There are many other domains that are vulnerable to Sybil attacks. It's impossible to enumerate all the domains susceptible to Sybil attack in 
this survey, so we have selected some typical domains that are well studied in section \ref{SybilAttacks}. If your domain is not listed in \ref{SybilAttacks}, 
that doesn't mean your domain is free from Sybil attacks. You should ask yourself if it's possible to create fake identities in your system, if your answer is yes then you should
keep an eye open for Sybil attacks.

The following is a roadmap of the rest of the paper.
In section \ref{SybilAttacks}, we define some common properties of sybil and a list of some major domains that are susceptible to Sybil attacks are provided.
We define the goal of the defense and three main types of defense mechanisms in section \ref{SybilDefense}. 
The three categories of Sybil defenses are traditional defense, social
network based defense and domain specific defense. We provide descriptions on how the defense works and point our their advantage/disadvantage
in terms of their efficiency, false positive/negative rate, deploy-ability and more. In the last section \ref{Conclusion}, we would conclude this survey.
After reading this survey you will have a deeper understanding of Sybil attacks and should be equipped with many techniques to defense against Sybil attacks.





% no \IEEEPARstart
\section{Attack Model}
The context of the attack can be vastly different. Sybil attacks can take place in vastly different domains such as Wireless Sensor Network, Online Social Network, 
Reputation System, Ad hoc Mobile Network and etc. The goal of the attacker can vary too. The goal of an attacker can either be to control the system for self benefit
or to subvert the normal functionality of the system. Attackers with the goal to manipulate the system will create sybil nodes that camouflage themselves as honest nodes and
acts like honest nodes. For example, in an Online Social Network, the attacker can gradually create sybils and make them looks like real users by using other people's
online profile and daily posts etc. Some attackers would go further to ensure their fake accounts act like regular users with regular logins, friend requests,
friend request acceptances and real user like click streams. After the sybils have made enough connection with honest users, they are start spreading news or malware 
to disrupt the targeted online social network. If the goal of the attacker is to subvert the targeted system, he/she will usually inject as many bad players into the system
as possible. Sybils are like bombs that are being hide in the system and when they explode at once, the system could be destroyed.

In sum, attackers can create sybils quickly or gradually. Sybils can have a short or long camouflage periods when they act as honest players.
Sybils can launch the attack all at once or they can misbehave one at a time. Sybils can have no connection within themselves or they can form relationship between each other
to form a group.

\section{Sybil Attacks} \label{SybilAttacks}
\subsection{Routing System}
Routing is an essential part of all distributed systems. Many P2P systems use DHT to store routing tables.
There are two major strategies to perform routing table attacks in such P2P systems named horizontal attack and vertical attack according to \cite{wang12dht}.
Assume a setting of Kademlia-based system\cite{maymounkov02kademlia} where the system has N nodes, each node maintains a k-bucket routing table and
the average number of hops in routing a message in the system is $O(log(N))$.

A horizontal attack aims at polluting as many routing tables as possible through spreading sybils widely across the whole system. 
Based on the k-bucket mechanism, an attacker need only to control at least one sybil among a node's neighbours to effectively intercept messages.
To launch a successful horizontal attack, an attacker would need to roughly inject $\frac{N}{min(k,O(log(N)))}$ sybils into the system to hijack the whole system.
The next question concern the attack is how much resource is needed to perform this attack. A straightforward way to do it is to run a sybil instance in one machine
but this is highly inefficient. By modifying the DHT client, attackers can run many sybil instances simultaneously on a machine. Moreover, by exploiting the hopping technique
in \cite{wolchok09defeatingvanish}, sybils can change their id periodically and jump to a new location in the DHT. By jumping into new locations periodically, a sybil instance
can cover a lot more area than a static sybil.

On the other hand, vertical attack tries to insert as many sybils as possible into one specific routing table. Using vertical attack, a specific content ID can be targeted.
This attack would be made more difficult with a very large DHT and if the DHT protocol assign random ids to newly join nodes but this is not the case in Mainline DHT(MLDHT)\cite{wang12dht}.
MLDHT allows nodes to pick their own IDs and this security weakness has been around for over a decade and no one care enough to fix it. 
Given a target Id, sybils can generate ids that will locate them close
to the target and 'isolate' the target from the others.

The authors in \cite{wang12dht} also mention a hybrid approach in which attacks would first launch a horizontal attack to take control over the whole system then target individual nodes
with vertical attack. In sum, a hybrid attack would lead to the attacker controlling the whole system.

There are other routing protocols that are vulnerable to Sybil attack.
Geographical routing protocol requires nodes to exchange coordinate data with their neighbours to efficiently address packets. 
By using Sybil attack, an attacker can create multiple identities in different geographical locations thus making him available in multiple places at once which violates the 
fundamental assumption of the geographical routing protocol\cite{Karlof03securerouting}. 
Sybil attack poses a thread to the seemingly robust multipath routing protocol too. For more detail please see \cite{Karlof03securerouting}.


\subsection{Content Rating System}
Sybil attack is a fundamental thread to many user-based content rating system such as GoodReads, YouTube and IMDB. 
There are huge incentives in this kind of attacks because attackers can promote low-quality content to a wide audience. 
It has been studied that many people check the IMDB score before going to the movie theatre. 
A high IMDB score will attract more audiences thus making the movie more profitable. This is not hypothetical. 
There are successful real world cases. 
For example, the famous Slashdot poll on the best computer science school has caused students to write automatic scripts to vote for their schools repeatedly. 
Moreover, some underground companies made money through assisting clients in promoting their YouTube video's view counts by using a large number of Sybil accounts.\cite{Tran09SOC}

\subsection{Online Market Place}
Sybil attack has posted a significant challenge for building reputation systems in online market place. 
In a reputation system, an adversary can create a large number of identities and maliciously increase the reputation of one or more master identities by giving false 
recommendations to them. 
Sybils can also promote their own reputations and falsely accuse well-behaved players in the system to hurt their reputation. 
For example, in eBay.com reputation is calculated as the sum of (+1,0.-1) of all the transaction ratings no matter how big the transaction is. 
Sybils can be create to make small transactions with a seller and automatically give them good reviews to boost their reputation. 
Afterwards, the seller can use that reputation on a dishonest transaction of high value. 
By using Sybil attack, a dishonest seller can hide the fact he frequently misbehaves at a certain rate.

Moreover, in networks that use reputation scheme to find misbehaving nodes/sybils, 
nodes with good reputation can report nodes they believe to be misbehaving in its neighbours. 
But this scheme can backfire. 
For example, users can collude to artificially boost the reputation values of one or more friends, or falsely accuse well-behaved users of misbehaviour. 
When adversaries control enough sybil nodes and decide to repeatedly report honest nodes. 
The outcome is that most of the honest nodes will be considered malicious and be removed from the networks, 
the malicious nodes will take full control of the whole system and use it for their own benefits. 
Detecting such collusion attacks is yet an unsolved problem that severely limits the impact of existing reputation systems.\cite{Swamynathan10reputation}\cite{Lian07anempirical}


\subsection{Resource Sharing System}
Sybil attack can be used to gain a disproportional share of resources in P2P network. 
In a P2P system, people can share their resources such as bandwidth, memory, computation power and file. 
An adversary can create sybils to claim an unfair and disproportionate share of the resources that were intended to be 
distributed amongst all nodes in the system. For example, in a public cloud infrastructure like Amazon Web Services, Google Cloud, 
each user is eligible for a free 15GB of data storage. An attacker can create 100 or more sybil accounts and claim more than 1500GB of free storage.
Let's consider a distributed file sharing system, the download speed depends on how much credit a user. To obtain credits, a user needs to make contribution and upload files to others.
This seems like a fair file sharing system until an attacker creates many sybil accounts and generate credits from download/upload files between themselves.
With Sybil attack, the attacker can generate unlimited credits and use them to quickly download files from others.


\subsection{Distributed Storage System}
In distributed storage system, nodes are required to score a fragment of a file and each fragment is duplicated into multiple machines to prevent data lost 
and increase the performance of file download speed.
Sybils can cause data lost by being selfish and not storing the fragment of data that they are asked to store. 
Sybils can also degrade the performance of the distributed file system by not responding to file request or provide the wrong file segment. 
What's more, because some distributed file systems replicate data to neighbouring nodes, sybils can be used to crawl the entire file system through frequently 
hopping to different locations in the network and obtain data fragments from all its neighbours.\cite{Lian07anempirical}\cite{wolchok09defeatingvanish}

\subsection{Online Social Network}
Online Social Networks(OSN) like Facebook and Twitter are vulnerable to Sybil attack as well. One goal of the attacker can be to crawl the OSN's user's personal data.
Personal data includes name, phone number, age, address etc. In order to obtain those personal information
attackers need to be friends with the actual users to be able to see those information.
An attacker in this case will create a lot of fake accounts that camouflage themselves as real users. New sybil accounts can be created by copying the information of some of the victims
who befriended a sybil account and have their personal information stolen. New sybils will continue to send friend requests to other people and some will even befriend themselves.
After using sybils to crawl users' account information, attackers can then make a profit from selling those information. Another goal of attacking the OSN can be to spread spam.
More and more people these days use facebook and twitter as their news channel and read posts on the Social network as they read news paper. 
After successfully befriend many users and their friends, a sybil account can more effectively spread spam or even malicious files.

\subsection{Collaborative Mobil Application}
People these days have spent more time on portable devices such as their smart phone or tablet than on their computers.
Researchers have recently proposed general infrastructures for portable devices within proximity of each other to trade various services
in a scalable and decentralize way without going through any Internet server. Collaborating devices can synchronize their times, run localization algorithms 
that increase the precision of street map software, borrow each other's bandwidth, or even share cached web content. The problem with this model is that it can 
easily be disrupted by uncooperative and malicious sybils. Those who only want to profit from these services and not providing anything in return.
They can usually create a number of sybil identities to avoid being tracked down since no identity certification authority is involved in this kind of model.\cite{quercia10mobile}

\section{Sybil Defense} \label{SybilDefense}
\subsubsection{Defense Goal}
Before going into the details of sybil defenses, the defense goal should be defined. It should be obvious that the ideal goal is to eliminate Sybil attack
but this goal might not be realistic deal to the fact that we don't want to enforce a strict rule whenever people join the our system.
Without a centralized trusted certification based scheme, the best we can do is to restrict the effects of sybils. False positive rate and false negative rate 
can serve as good metrics when evaluating a defense approach. False positive happens when a honest node is identified as sybil using the defense mechanism and
false negative happens when a sybil node is not detected and treated as a honest user. So the goal in the following defense mechanisms is to minimize the false 
positive and false negative rates as much as possible.

\subsubsection{Classification of Defenses}
There are general defense mechanisms that work in most domains and there are domain specific approaches that target a specific domain and works better in that domain by levering some 
domain specific features. We further divide the general approach into traditional approach and social network based approach. 
Traditional approach were approaches found between 1999 to 2004. They focus mainly on building identity authentication, resource testing or human assisted sybil detection mechanisms
 into existing systems. On the other hand, social network based approach started around 2006 tries to detect sybils by exploiting key social network structures underlying the system.

\section{General Approach}
There are two types of general approaches. The traditional approach and the social network based approach. 

\subsection{Traditional Approach}
Traditional approach is based on research starting around 1999 and end around 2004. During that period of time, researchers have focused on preventing Sybil attacks by involving secure mechanisms such as digital signatures and identity authentication. Other methods have also been found to increase the resource cost of a Sybil attacks such as resource testing and recurring cost. Moreover, the human assisted sybil detection approach has been widely deployed during that time.

\subsubsection{Trusted Certification}
This is the most popular solution for countering Sybil attacks, it required a trusted certifying authority that validates the identity of a node before it joins the system. There are two variations in this approach. One is the centralized version, the other is the semi-centralized version. In the centralized version, it is assumped that there is a trusted central authority who can verify the validity of each participant. After the validation, a certificate will be given to each participant. The participant then can use the certificate to access the system. The model is very popular and has been used widely. Most authentication services use this kind of model. The semi-centralized approach seek to cut of the cost of asymmetric criptography used in the centralized version. It leverage a techniqe called partical identity verifications. The approach still need to rely on a trusted base station but reduce the involvement of a third party authority.

The problem of  trusted certification approach is that it rely on a centralized trusted authority for credential generation, assignment and verification. 
However, it sacrifice the open nature that underlies the success of these distributed systems and increase the overhead of the system. 
\cite{newsome04sybil}\cite{Castro02Secure}\cite{Adya02FFA}.

\subsubsection{Resource Testing and Recurring Cost}
Resource Testing is another line of solution. The idea behind resource testing is that each identity should own a fair amount of resource because it runs on a 
legitimate client otherwise there is a high potential that this is a sybil node. The question is how can we test that there are resource backing up a node? 
Some propose the testing of IP address because multiple identities sharing a single IP address is a good sign of Sybil attacks. 
Others test resources such as computing power, network bandwidth, MAC address. This approach in theory should work for systems that are in very low risk. 
It’s easy to implement resource testing or recurring cost approaches but people these days can acquire a large amount of resources in a short period of time with the help of public cloud.
Resource testing is mostly obsolete when an attacker can spin of hundreds of EC2 instances in a short period of time and terminate them after a few hours of attack. 
As of this writing, the EC2 t2.nano instance will only cost \$0.0043 per hour with upfront payment.


A variation of the resource testing method is called “Recurring Costs”. For example, in one solution participants are required to perform some tasks such as solving puzzles\cite{Borisov06CPS} periodically. The biggest disadvantage of computational puzzle is that it will prevent honest users with old computers from joining the system. Turing tests like CAPTCHA are also suggested as a recurring cost solution\cite{Ahn03CUH}. Using this approach, the cost of Sybil attacks have become more expensive but would the benefit still outweighs the cost? This approach is not recommended in high risk system for the following reasons. Computational puzzle can hold back entry level attacks but not advance attacks that leverage the public cloud’s computing power. How about turing tests? Is it not a lot of Online Social Network sites still use this approach? With crowdsourcing services like Amazon Mechanical Turk, turing tests can be crowdsourced at a reasonably low price.

\subsubsection{Human Assisted Approach}
We believe the human assisted approach has been the oldest sybil defense approach ever. 
This approach has been used to fight against identity theft. Even today, there is still no automatic way to identify fake identity. 
Online social network sites have long been using this approach to find fake accounts due the the failure in automated fake account detection. 
This approach starts with the honest users report a potential fake accounts. 
Afterwards, an account police start the manual inspection involving matching profile photos to the age or address, understanding natural language in posts, 
examining the friends of the user, etc. 
This approach is time consuming. 
Tuenti a Spanish social network receives on average 12,000 reports regarding fake accounts per day and only about 5\% of them are indeed fake. 
This approach can effectively offload some of the sybil detection work to its honest users but should be kept to its minimum by combining it with another 
automated sybil detection technique.

\subsection{Social Network-Based}
Yu et al. has started a new era of sybil defense when he proposed the idea of detecting sybils using a unique structure in the social network graph. Even though attackers can inject many sybils into a social graph, the connections between honest users and sybils are limited\cite{Yu08SybilGuard}. For example, honest users on facebook would not randomly accept friends if they do not know the person. Suprisingly, the social network appraoch has showed to be able to overcome some of the earlier approaches limitations and shortcomings.
\subsubsection{SybilGuard}
SybilGuard designed by Yu et al. \cite{Yu08SybilGuard} is one of the first Sybil defense techniques based on Social Network. The approach assumes that each edge in the graph between two identities indicates a human-established trust relationship and malicious users can only create limited edges between honest users. SybilGuard bounds the number of malicious sybils a user can create by exploiting the property that there exist a disproportionaltely small "cut" in the 
graph between the sybil nodes and the honest nodes.

\subsubsection{SybilLimit}
The approach take by SybilLimit in \cite{Yu08SybilLimit} is the same as SybilGuard but SybilGuard can dramatically reduce the number of sybil nodes accepted by a factor of  $ \sqrt{n} $.

\subsubsection{SybilInfer}
SybilInfer takes the approach of labelling nodes in a social network as honest users or Sybils. Internally, it uses a probabilistic model of honest social networks as its knowledge base and an inference engine to obtain the potential regions of dishonest nodes.
It claims to be more accurate and more applicable when compare to both SybilGuard and SybilLimit.


\section{Project Deliverable}
% no \IEEEPARstart
We have listed potential threads of Sybil attacks under different context and showed different types of counter measurements. 
We want the final paper to serve as introduction and guidline for sybil attacks and defenses.
To obtain this goal, we would like to present more detail explaintion of each Sybil defense approaches in the final survey. Furthermore, a comphrehensive comparison among the different defense mechanisms will be included.
If time permits, we would also look into evaluating some of the defense mechanisms by using metrics like false negative/positive rate, the time and code complexity the soution would add to the existing system.
Also we would like to point out potential directions/opportunities in the research of sybil attacks/defenses in the final survey paper.

% An example of a floating figure using the graphicx package.
% Note that \label must occur AFTER (or within) \caption.
% For figures, \caption should occur after the \includegraphics.
% Note that IEEEtran v1.7 and later has special internal code that
% is designed to preserve the operation of \label within \caption
% even when the captionsoff option is in effect. However, because
% of issues like this, it may be the safest practice to put all your
% \label just after \caption rather than within \caption{}.
%
% Reminder: the "draftcls" or "draftclsnofoot", not "draft", class
% option should be used if it is desired that the figures are to be
% displayed while in draft mode.
%
%\begin{figure}[!t]
%\centering
%\includegraphics[width=2.5in]{myfigure}
% where an .eps filename suffix will be assumed under latex, 
% and a .pdf suffix will be assumed for pdflatex; or what has been declared
% via \DeclareGraphicsExtensions.
%\caption{Simulation results for the network.}
%\label{fig_sim}
%\end{figure}

% Note that the IEEE typically puts floats only at the top, even when this
% results in a large percentage of a column being occupied by floats.


% An example of a double column floating figure using two subfigures.
% (The subfig.sty package must be loaded for this to work.)
% The subfigure \label commands are set within each subfloat command,
% and the \label for the overall figure must come after \caption.
% \hfil is used as a separator to get equal spacing.
% Watch out that the combined width of all the subfigures on a 
% line do not exceed the text width or a line break will occur.
%
%\begin{figure*}[!t]
%\centering
%\subfloat[Case I]{\includegraphics[width=2.5in]{box}%
%\label{fig_first_case}}
%\hfil
%\subfloat[Case II]{\includegraphics[width=2.5in]{box}%
%\label{fig_second_case}}
%\caption{Simulation results for the network.}
%\label{fig_sim}
%\end{figure*}
%
% Note that often IEEE papers with subfigures do not employ subfigure
% captions (using the optional argument to \subfloat[]), but instead will
% reference/describe all of them (a), (b), etc., within the main caption.
% Be aware that for subfig.sty to generate the (a), (b), etc., subfigure
% labels, the optional argument to \subfloat must be present. If a
% subcaption is not desired, just leave its contents blank,
% e.g., \subfloat[].


% An example of a floating table. Note that, for IEEE style tables, the
% \caption command should come BEFORE the table and, given that table
% captions serve much like titles, are usually capitalized except for words
% such as a, an, and, as, at, but, by, for, in, nor, of, on, or, the, to
% and up, which are usually not capitalized unless they are the first or
% last word of the caption. Table text will default to \footnotesize as
% the IEEE normally uses this smaller font for tables.
% The \label must come after \caption as always.
%
%\begin{table}[!t]
%% increase table row spacing, adjust to taste
%\renewcommand{\arraystretch}{1.3}
% if using array.sty, it might be a good idea to tweak the value of
% \extrarowheight as needed to properly center the text within the cells
%\caption{An Example of a Table}
%\label{table_example}
%\centering
%% Some packages, such as MDW tools, offer better commands for making tables
%% than the plain LaTeX2e tabular which is used here.
%\begin{tabular}{|c||c|}
%\hline
%One & Two\\
%\hline
%Three & Four\\
%\hline
%\end{tabular}
%\end{table}


% Note that the IEEE does not put floats in the very first column
% - or typically anywhere on the first page for that matter. Also,
% in-text middle ("here") positioning is typically not used, but it
% is allowed and encouraged for Computer Society conferences (but
% not Computer Society journals). Most IEEE journals/conferences use
% top floats exclusively. 
% Note that, LaTeX2e, unlike IEEE journals/conferences, places
% footnotes above bottom floats. This can be corrected via the
% \fnbelowfloat command of the stfloats package.


% trigger a \newpage just before the given reference
% number - used to balance the columns on the last page
% adjust value as needed - may need to be readjusted if
% the document is modified later
%\IEEEtriggeratref{8}
% The "triggered" command can be changed if desired:
%\IEEEtriggercmd{\enlargethispage{-5in}}

% references section

% can use a bibliography generated by BibTeX as a .bbl file
% BibTeX documentation can be easily obtained at:
% http://mirror.ctan.org/biblio/bibtex/contrib/doc/
% The IEEEtran BibTeX style support page is at:
% http://www.michaelshell.org/tex/ieeetran/bibtex/
%\bibliographystyle{IEEEtran}
% argument is your BibTeX string definitions and bibliography database(s)
%\bibliography{IEEEabrv,../bib/paper}
%
% <OR> manually copy in the resultant .bbl file
% set second argument of \begin to the number of references
% (used to reserve space for the reference number labels box)

\bibliography{diy} 
\bibliographystyle{ieeetr}


% that's all folks
\end{document}


